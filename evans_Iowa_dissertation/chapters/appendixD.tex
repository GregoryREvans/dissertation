\chapter{Akasha Sketches}
\label{AppendixD}

\lettrine[lines=2,slope=-2pt,nindent=2pt]{\textcolor{SchoolColor}{I}}{n the following appendix} I have collected all of the available sketch materials found in the repository for Akasha's software implementation found at \url{https://github.com/trevorbaca/akasha}.

\section{Materials Document A}

\begin{figure}[H] % try H or h or p?
    % \centering
    \includegraphics[scale=0.65, page=1, angle=90]{akasha_sketches/01a-materials.pdf}
    \caption{Akasha Materials Page 1}
    \label{fig:akashamaterialsketch-1}
\end{figure}

\begin{figure}[H] % try H or h p?
    % \centering
    \includegraphics[scale=0.7, page=2, angle=90]{akasha_sketches/01a-materials.pdf}
    \caption{Akasha Materials Page 2}
    \label{fig:akashamaterialsketch-2}
\end{figure}

\begin{figure}[H] % try H or h p?
    % \centering
    \includegraphics[scale=0.7, page=3, angle=90]{akasha_sketches/01a-materials.pdf}
    \caption{Akasha Materials Page 3}
    \label{fig:akashamaterialsketch-3}
\end{figure}

\begin{figure}[H] % try H or h p?
    % \centering
    \includegraphics[scale=0.7, page=4, angle=90]{akasha_sketches/01a-materials.pdf}
    \caption{Akasha Materials Page 4}
    \label{fig:akashamaterialsketch-4}
\end{figure}

\begin{figure}[H] % try H or h p?
    % \centering
    \includegraphics[scale=0.7, page=5, angle=90]{akasha_sketches/01a-materials.pdf}
    \caption{Akasha Materials Page 5}
    \label{fig:akashamaterialsketch-5}
\end{figure}

\section{Materials Document B}

DESIDERATA
==========

1. wonder

2. breath; halting

3. consonance

4. layered time

5. memory

MATERIALS
=========

1. 1/2 clt

2. multiinstrument tremolo flautando

3. multiinstrument natural harmonic tremolo

4. tasto triple-muting

5. stuttering exposition

7. individuated clicks

8. tutti hairpins

9. low

10. true ppppp

COLORS
======

A. getati\\
\quad  a1. isolated scratch tenuti\\
\quad  a2. rhythmically irregular scratch detache\\
\quad  a3. mf getato pos. ord\\
\quad  a4. dense getato piano\\
\quad  a5. ext. dense getato pianiss.\\
\quad  a6. max. dense getate pianiss.\\
\quad  a7. perforated getato pianiss.\\
\quad  a8. very perf. getato pianiss.\\
\quad  a9. silence with occ. getato strands\\

B. microtonal field\\
\quad  b1. solo microtonal melodic line on accel / rit with SB\\
\quad  b2. two-voice microtonal polyphony char. mod. rhythm pSB\\
\quad  b3. four-voice microtonal polyphony char. lento rhythm SB\\
\quad  b4. fixed 8-note 12ET chord SB > ord\\
\quad  b5. fixed 8-note 24ET chord ord > FB\\
\quad  b6. fixed 8-note 12ET chord FB > XFB\\
\quad  b7. fixed 8-note 24ET chord XFB\\
\quad  b8. reartic. 8-note 12ET chord pSB lento rit. dim.\\
\quad  b9. reartic. 8-note 12ET chord pSB lentiss. rit. dim.\\
\quad  b10. fixed 8-note 12ET chord SB dim.\\

C. massed trills\\
\quad  c1. isolated tasto tenuti or trills\\
\quad  c2. isolated 2- or 3-note tenuti or trills\\
\quad  c3. 3-note tenuti or trill phrases with intermitt. breaks\\
\quad  c4. 3-note tenuti / trill loops w/o breaks mixed mecc. fluido\\
\quad  c5. 3-note tenuti / trill loops w/o breaks mecc. fluido block dynamics\\
\quad  c6. fixed trill field with t > xsp > t enveloping \\
\quad  c7. xsp swipe against four-note root chord\\

D. spectral field\\
\quad  d1. vc, va detuned octave\\
\quad  d2. tutti detuned octaves\\
\quad  d3. vc, va arco 9-7\\
\quad  d4. tutti arco 9-7-6-5\\
\quad  d5. tutti A9-A7-A6-A5 trill field\\
\quad  d6. tutti A9-A7-A6-A5 trill field with t > xsp > t enveloping\\
\quad  d7. xsp swipe against 9-7-6-5 spectral chord\\

E. OB\\
\quad  e1. fixed OB\\
\quad  e2. quarter note OB\\
\quad  e3. quarter note 3/4 OB, 1/2 OB\\
\quad  e4. quarter note 1/4 OB, XSP\\
\quad  e5. quarter note stopped XSP\\
\quad  e6. quarter note open-string XSP\\
\quad  e7. quarter note ppp glissandi XSP\\

\section{Moments Document}

\begin{figure}[H] % try H or h p?
    % \centering
    \includegraphics[scale=0.65]{akasha_sketches/02-moments.pdf}
    \caption{Akasha Moments}
    \label{fig:akashamomentsketch}
\end{figure}

\section{Stages Document}

\begin{figure}[H] % try H or h p?
    % \centering
    \includegraphics[scale=0.6, page=1]{akasha_sketches/akasha_stages.pdf}
    \caption{Akasha Stages Page 1}
    \label{fig:akashastagessketch-1}
\end{figure}

\begin{figure}[H] % try H or h p?
    % \centering
    \includegraphics[scale=0.65, page=2]{akasha_sketches/akasha_stages.pdf}
    \caption{Akasha Stages Page 2}
    \label{fig:akashastagessketch-2}
\end{figure}

\begin{figure}[H] % try H or h p?
    % \centering
    \includegraphics[scale=0.65, page=3]{akasha_sketches/akasha_stages.pdf}
    \caption{Akasha Stages Page 3}
    \label{fig:akashastagessketch-3}
\end{figure}

\begin{figure}[H] % try H or h p?
    % \centering
    \includegraphics[scale=0.65, page=4]{akasha_sketches/akasha_stages.pdf}
    \caption{Akasha Stages Page 4}
    \label{fig:akashastagessketch-4}
\end{figure}

\begin{figure}[H] % try H or h p?
    % \centering
    \includegraphics[scale=0.65, page=5]{akasha_sketches/akasha_stages.pdf}
    \caption{Akasha Stages Page 5}
    \label{fig:akashastagessketch-5}
\end{figure}

\begin{figure}[H] % try H or h p?
    % \centering
    \includegraphics[scale=0.65, page=6]{akasha_sketches/akasha_stages.pdf}
    \caption{Akasha Stages Page 6}
    \label{fig:akashastagessketch-6}
\end{figure}

\begin{figure}[H] % try H or h p?
    % \centering
    \includegraphics[scale=0.65, page=7]{akasha_sketches/akasha_stages.pdf}
    \caption{Akasha Stages Page 7}
    \label{fig:akashastagessketch-7}
\end{figure}

\begin{figure}[H] % try H or h p?
    % \centering
    \includegraphics[scale=0.65, page=8]{akasha_sketches/akasha_stages.pdf}
    \caption{Akasha Stages Page 8}
    \label{fig:akashastagessketch-8}
\end{figure}

\begin{figure}[H] % try H or h p?
    % \centering
    \includegraphics[scale=0.65, page=9]{akasha_sketches/akasha_stages.pdf}
    \caption{Akasha Stages Page 9}
    \label{fig:akashastagessketch-9}
\end{figure}

\begin{figure}[H] % try H or h p?
    % \centering
    \includegraphics[scale=0.65, page=10]{akasha_sketches/akasha_stages.pdf}
    \caption{Akasha Stages Page 10}
    \label{fig:akashastagessketch-10}
\end{figure}

\begin{figure}[H] % try H or h p?
    % \centering
    \includegraphics[scale=0.65, page=11]{akasha_sketches/akasha_stages.pdf}
    \caption{Akasha Stages Page 11}
    \label{fig:akashastagessketch-11}
\end{figure}

\begin{figure}[H] % try H or h p?
    % \centering
    \includegraphics[scale=0.65, page=12]{akasha_sketches/akasha_stages.pdf}
    \caption{Akasha Stages Page 12}
    \label{fig:akashastagessketch-12}
\end{figure}

\begin{figure}[H] % try H or h p?
    % \centering
    \includegraphics[scale=0.65, page=13]{akasha_sketches/akasha_stages.pdf}
    \caption{Akasha Stages Page 13}
    \label{fig:akashastagessketch-13}
\end{figure}

\begin{figure}[H] % try H or h p?
    % \centering
    \includegraphics[scale=0.65, page=14]{akasha_sketches/akasha_stages.pdf}
    \caption{Akasha Stages Page 14}
    \label{fig:akashastagessketch-14}
\end{figure}

\section{Program Notes}

The inscription at the head of the score reads as follows:

\begin{quote}
    \singlespacing
    Akasha is a music of invisibility, electricity and the open expanse of the sky. The title is the Sanskrit word for a concept once understood as an unseen force present in all things in motion in the world.
\end{quote}

Written for string quartet scordatura, Akasha suggests an elaborate blossoming-forth of different types of thinking — different ways of knowing — magically experienced all at the same time. Scored in a single movement lasting half an hour, the constellation of ways the music starts, stops, splinters and recombines suggests the shapes of thought. The title of the piece — variously translated æther or quintessence — might also be understood as pointing to the vivid experience of indirect motion that comes to us in music when we experience it with both intensity and joy.

Alternating in brilliance and grit, the musical moments in Akasha derive from an inventory of just five materials. The starting forms of these materials — leggierissimo flurries of staccati played off-string; two-, three- and four- voice polyphony admixed with noise; broken ritardandi and accelerandi played with characteristic flautando; octavated harmonic fields reinforced below the thirteenth partial; and the white continuity of the bows’ hair drawn directly across the wood of instruments’ bridges — develop according to an eleven-part regime of transformations given four times in succession. The resulting series of forty-four material combinations supplies the music’s middle structure. This structure is then reinterpreted as a protracted succession of musical moves repartitioned into the seventy-one rest-delimited moments apparent in the piece.

Dramatic moments in the music insist on repetition in unanticipated ways. The pair of ferocious climaxes, or vortices, near the end endure unexpectedly long in their sixfold echoes of moments hidden earlier in the piece. These vortices neither depict experience nor metaphorize reality. The vortices stand instead in a relation to a one- pointedness of thought that considers itself. This thought that considers itself is a sensualized experience of time that discovers the electrified edges of its shapes amid an experience of its own duration. How we experience the effects of these shapes depends on who we are in the world at the moments of the shapes’ arrival. In the ponticello-to-midpoint transitions of the climaxes — and perhaps at earlier moments, too — who we are as we listen is perhaps changed: an electrified waters of forgetfulness that covers who we thought we may have been.

The many rests intercalated between Akasha’s moments carry fermatas — short, medium, long, very long — that divide the music perceptually. The longer classes of fermata, in particular, ask the players for unusually protracted durations that are long enough to cause us to just barely begin to forget the course of the music in the middle of the course of the music. Doing this effectively requires a very good string quartet and an ability to think differently about the ways that both music and events in the world start and stop: the interpenetration of the music’s silences perhaps provides us the opportunity to hear stops not as endings but as events swept up in currents that carry away starts and stops alike.

As moments of the music transform from shorter to longer — and from disjointed to insistent — it is possible to experience the music according to a metaphor of space. But the pervasiveness of something like the garden metaphor (“I am walking in a garden; I recross my paths; `objects’ reappear according to different points of view”) might best be suspended here. The moments manifest in Akasha’s flow are not conceived in the first instance as objects subject to a point of view. The moments in this music aspire to the movement of carriers freed from space in their role as bringers of this-wordliness in our experience of thought.

\phantom{text} \hfill — Trevor Bača

\section{Dissertation Commentary}

{\hindifont आकाश} (Akasha) was written in 2015 for the JACK Quartet who gave the premiere in February 2016 in Paine Hall at Harvard University. Like the bass clarinet solo written before it, \textit{Akasha} proceeds according to a formal polyphony derived from the juxtaposition, alternation and superimposition of five musically independent narratives precomposed before the determination of the in-time details of the score. The formal experience of the piece underlines the role played by memory: cases of exact recurrence serve to problematize, rather than reinforce, the narrative experience of the music -- the unexplained d\'eja v\`u of the opening five-note cello melody played half-scratch, for example -- while incessant varied recurrence of material drives the music’s development. Sections appear in widely different durations: from the time of a single note to more than three minutes’ rearticulation of a chordal field. A crucial role is played by the distribution of the music’s silences, saturating the piecemeal exposition of the piece and becoming progressively less prevalent in the quickening formal experience produced as the music progresses. Like \textit{Krummzeit}, written the year before, the deployment of material in \textit{Akasha} effects both a dilation and contraction of time seemingly at once, mixing narrative and even fantastical senses of time with the recurrent phenomenological experience of the arrival and departure of musical materials considered in themselves. Three sixfold repetition vortices appear in the music. The first of these, at the end of \boxed{\text{F}}, is rendered in incredibly quick leggierissimo notes played off-string in all four parts at once; the difficulty of the music ensures not only that each of the six passes differs from the others but that we experience each of the six passes as different to the others: the panta rhei -- the `everything flows' -- argued in details of the music’s material. The second and third repetition vortices, in \boxed{\text{K}}, comprise the outsize climaxes of the piece: first in the form of a four-octave chordal field scored in equal temperament and just intonation at the same time and then in the form of the electricity of harmonic trills anchored to the eleventh partial of the cellos down-tuned lowest string. The experience of these moments of repetition -- looking back at comparable passages in \textit{Krummzeit}, \textit{Ins Wasser eingeschrieben} and \textit{\v{C}\'ary} -- summons a magic to transfix perception on the invisible energies that exist at a behind of matter or, perhaps, to look directly at the sun.\footnote{\citet{baca-dissertation}}

\section{Annotated Sections of Akasha}

\includepdf[pages=-, angle=90, pagecommand={},
  width=\textwidth, keepaspectratio]{lilypond/akasha_01.pdf}

\includepdf[pages=-, angle=90, pagecommand={},
  height=\textheight, keepaspectratio]{lilypond/akasha_02.pdf}

\includepdf[pages=-, angle=90, pagecommand={},
  height=\textheight, keepaspectratio]{lilypond/akasha_03.pdf}

\includepdf[pages=-, angle=90, pagecommand={},
  height=\textheight, keepaspectratio]{lilypond/akasha_04.pdf}

\includepdf[pages=-, angle=90, pagecommand={},
  height=\textheight, keepaspectratio]{lilypond/akasha_05.pdf}

\includepdf[pages=-, angle=90, pagecommand={},
  height=\textheight, keepaspectratio]{lilypond/akasha_06.pdf}