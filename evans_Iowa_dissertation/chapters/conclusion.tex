\begin{savequote}[75mm]
The treasure chest of the frequently thorny achievements of concert music after WWII, of musical modernism, is a wonder where time flows in impossible directions, sounds appear from instruments like magic, and musical logic is free to bewilder. [...] What's needed now, what's been missing for decades, is a musical answer to the magic realisms of the 20th- and 21st-century [...]. What is needed is an engagement with musical inheritance that finds space to accommodate beauty, difficulty, and the transfiguration of the one by the other, since that seems, inescapably, to be the world we live in.
\qauthor{Trevor Bača\footnote{\citet{harmonyscore}}}
\end{savequote}

\chapter{Conclusion}
\label{conclusion}

% \lettrine[lines=2,slope=-2pt,nindent=2pt]{\textcolor{SchoolColor}{I}}{n this dissertation}

\lettrine[lines=3]{\setmainfont{GoudyInitialen}[Path=./fonts/, Extension = .ttf]\color{printGreen}I}{n this dissertation}, I have described the compositional techniques of three composers who have significantly influenced my compositional approach and I have shown certain aspects of their compositional process which are present within my own work. Undoubtedly, as a result of my own use of the Abjad \ac{API} for \ac{FSC}, my music conforms more closely to that of Trevor Bača, however I have my own criteria for the configuration of formal structures and generative streams of information. In recent works, such as 2019's \textit{(HARMONY)}, Bača composes more frequently through the direct input of rhythm and pitch materials, leaving generative process for the proliferation of more tedious tasks, for instance the mass accumulation of grace-note figures.

My work has been influenced by works which have a strong coupling of musical parameters to create definite statements of identifiable musical material. In the future, I expect further developments in my music will occur in the areas of the development of motivic and figural materials as opposed to purely textural materials, as well as an increased interest in subtly drawing cross-relationships between different materials. A possible avenue of change will also be a greater propensity for the admixture of material. Increased ambiguity in the cross-relationships between different materials can also potentially provide an interesting complexification of the narrative arc of a piece. In fact, I have already begun experimenting with this process in my work \textit{Infiorescenze} for solo alto flute.

\begin{figure}[p]
\centering
    \rotatebox{90}{
    \resizebox{0.9\textheight}{!}{
    \includegraphics[width=6in]{lilypond/infiorescenze_example.pdf}
    }
    }
    \caption{A page from Infiorescenze for solo alto flute}
    \label{fig:infiorescenze}
\end{figure}

I am also increasingly interested in providing extra-musical connections to the materials within my work. This can provide a fresh, new perspective on the underlying structure of a material. Hèctor Parra's string quartet \textit{Arachne} features a material reminiscent of a spider extruding and plucking a web. This material is not treated narratively in the sense of a tone-poem, rather the figure is deconstructed and developed motivically.\footnote{This example was explained to me by the composer during the 2023 MIXTUR Festival workshops in Barcelona, Spain.} I see great potential in the development of figural materials which can be manipulated multi-parametrically as opposed to the sometimes undifferentiated quality of texture generated through Bača's rhythm makers which have become so foundational to my own rhythmic approach.

Through the process of understanding the large-scale formal qualities of the works described in this dissertation, I have begun to taxonomize my experience of these events. I detect certain effects as being provided by specific types of transition techniques. Different types of repetition\footnote{Here I mean literal, immediate repetition by way of repeat bar lines rather than long-term recurrence relationships throughout the course of a work.} have different psychological effects in the memory timestream of my listening experience. While I am certain my experience of these formal moves and others which I have not mentioned are entirely subjective, I am encouraged to continue taxonomizing these event-types in order to help guide my future compositions.

\section{Artist's Statement}

The intersection of narrativity and abstract structuralism is where my creative efforts are focused. My imagination is sparked by a combined interest in linguistic structure, writing systems and their calligraphy, and alternative methods of expression, for example the cultures of the esoteric which can be seen in the texts of alchemy and magic. I believe in the power of newness, that objects once found repulsive can become beautiful and that old thoughts can be recontextualized and thus recuperated. I believe in the power of the act of decipherment, that hidden meaning, once decoded is somehow amplified above that which is plainly said. The shared experience of taking apart an encrypted work can bring a community into being through the collaborative effort of shared discovery.

I am interested in musical form as the direct activation of the experience of memory. My sense of formal technique is inspired by the filmic experience of time-cuts, flashes forward and back in time, the dilation of time resultant from differing velocities. I view composition as the act of giving life to sound as pleasurable intensity through impossible timelines, fantastic colors, and unusual shapes. Regions of stability and directed moments of the manifestation of newness exchange with one another, sometimes in rapid succession to illicit the impression of changes of scene in the state of an infinite “meanwhile.”

I am aware of many layers of transmutation in operation at once. My works are often the result of a period of intense formalization of large-scale structures and gestural patterns. A coded language of physical possibilities is converted into the symbolic language of music notation. This notation is then interpreted by performers as a series of actions latent with meaning which is later deployed in real-time with all the accidents and imperfections inherent in this practice before an audience who are tasked with the absorption and psycho-embodiment of the sounding experience.

Preparing the score carries the same value of intensity and emotional urgency as the process of imagining the intangible sound physicalities which appear during a composition. Creating the score is a symbolic act, the product of which may only suggest a series of potential meanings, however interpretations at all levels, analytic, emotional, or otherwise have the power, when in combination with one another, to grip, concentrate, and lead the audience to a transfiguration of the secret language of symbols. The shared public experience of music should bear the intensity of the knowing and the feeling of the things of our world.

In recent compositions, I begin at the large-scale and proceed by crafting finer and finer detail. I create a story in speeds, the map of tempi which will lead the work. I study the instrumental forces involved and partition out manners of sound production, grouping together similar sound palettes and contrasting accompaniments. From these potential sound qualities, I distill patterns of motion, operations of harmonic elaboration, rhythmic qualities, dynamic trajectories, and other factors of each narratively differentiated material type. The materials are then developed. How can a sound become another sound? How can one material type be made into the doppelgänger of another? I determine the approximate duration of the work and divide the totality into subsections and metrically divide the subsections by a pattern of time signatures. I produce a timeline graph of the exchange of materials, whether they overlap or abut, in which measures they occur. Then begins the task of specifying the details in score notation.