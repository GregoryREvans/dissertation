\begin{savequote}[75mm]
My freedom thus consists in my moving about within the narrow frame that I have assigned myself for each one of my undertakings. I shall go even further: my freedom will be so much the greater and more meaningful the more narrowly I limit my field of action and the more I surround myself with obstacles. Whatever diminishes constraint, diminishes strength. The more constraints one imposes, the more one frees one's self of the chains that shackle the spirit.
\qauthor{Igor Stravinsky\footnote{\citet[65]{stravinsky-quote}}}
\end{savequote}

\chapter{\textit{Sonate pour piano} (1952) by Jean Barraqué}%\chaptermark{TESTING}
\label{Chapter1a}
\lettrine[lines=3]{\setmainfont{GoudyInitialen}[Path=./fonts/, Extension = .ttf]\color{printGreen}D}{espite a small output} of six officially acknowledged works, partly a result of his early death,\marginpar{Barraqué's early death is not dissimilar to Francisco Guerrero's, both of which resulted in the diminishing of the performance of and scholarship around their works, particularly in the United States.} the music of Jean Barraqué (1928–1973) reveals a unique, confident musical voice investigating the dialectic relationship between lush, romantic lyricism and bombastic, violent dynamism. Standing in contrast to the other compositions under consideration in this dissertation, Jean Barraqué's 1952 \textit{Sonate pour piano} develops a binary form as opposed to the rhizomatic spaces of Guerrero and Bača. In his \textit{Sonate}, one of his earliest acknowledged works,\footnote{\citet[36]{barraque-griffiths}} Jean Barraqué develops a personal approach to multiple-serial composition. Due to this work's status as an already intensely analyzed composition, this chapter will only describe the work insofar as it relates to the understanding of material statements, event groups, and variegation.

Analysis of available sketch materials reveals an approach in which Barraqué defines multiple unique progressions, one for each parameter of rhythm, pitch, register octavation, articulations, and tempo as a means for creating a form of unique phases of activity. The similarity in approach between these materials, nameley serial ordering, hardly constitutes a reservoir of salient event groups; and while multiple parameters are manipulated simultaneously, a variegated sense of material polyphony is not produced.\marginpar{Since this work is composed for a solo instrument, despite it's polyphonic ability, it is perhaps unfair to negatively compare Barraqué's \emph{Sonate} with the variegated works explored later in this dissertation.} Nevertheless, a brief analysis of this work reveals a unique attempt to articulate large-scale form in a formalized context.

Thorough analysis of this score proves difficult due to the errors produced by the composer and by the editor of the original published edition. In this chapter, I will rely on the urtext edition prepared by Heribert Henrich.\footnote{\citet{barraque-score}} The editing process, explained in detail in the commentary volume published alongside the urtext score,\footnote{\citet{barraque-commentary}} was performed by comparing several manuscripts and resulted in the revision of some pitches and rhythm alignment. Henrich also performed revisions on some aspects of the tempo scheme. An exciting feature of this edition is the inclusion of proper measure numbers which can be used for the identification of the locations of various musical events.\marginpar{Past analysis of this work labeled measures by referencing the page number, system, measure, and sub-measure, resulting in some rather unwieldy measure indicators such as 28.1.1.3}

\section{Form}

The preface to the \textit{Sonate} describes a large binary form articulated primarily by the use of fast tempo versus slow tempo. Barraqué indicates that the second section begins on page 28 of the 1965 edition published by Firenze in the third measure of the third system, despite the introduction of slow tempo passages occurring slightly earlier. The urtext edition locates the beginning of the second part at measure 418 on page 33.

\begin{table}[H]
    \centering
    \begin{tabular}{l l | l l}
        & \textit{première partie:} & & \textit{deuxième partie:} \\
        \hline
        \textbf{A} & \textbf{Rapide} & \textbf{B} & \textbf{Lent}\\
         & Très rapide (début, style \guillemotleft libre\guillemotright) & & \\
         A I & Modéré & B I & Entre Modéré et Lent \\
         A II & Très Vif & B II & Large \\
         A III & Entre Modéré et Vif & B III & Très Lent \\
         A III* & Entre Modéré et Très Vif & &  \\
    \end{tabular}
    \caption{Tempo scales in \textit{Sonate pour piano} by Jean Barraqué}
    \label{tab:preface-tempo-table}
\end{table}

The tempo work in this piece forms several nested layers of manipulation. An outer layer of tempo dictates the basic speed, rubato is indicated in the next layer (see the indication \textit{Moins rapide} in measure 4), and one further layer of tempo manipulation is performed through indications of continuous tempo ramping such as \textit{accelerando} in measure 8, \textit{ritenuto} in measure 23, or \textit{presser} in measure 24. Table \ref{tab:first-tempo-sequence} shows the sequence of tempo changes of the primary layer from measures 1-183. The majority of tempo indications in the first part of the sonata do, in fact, correspond to those found in reservoir A and those in the second part correspond to tempi found in reservoir B. Occasionally a tempo appears which differs from the surrounding tempo context as a form of contrast.

\begin{table}[p]
    \centering
    \begin{tabular}{c | l}
        \textit{measure} & \textit{tempo} \\
        \hline
        1 & Très rapide \\
        8 & B \\
        39 & Très rapide \\
        55 & A II \\
        56 & A I \\
        57 & A II \\
        59 & A I \\
        60 & A II \\
        61 & A I \\
        62 & A III \\
        67 & A II \\
        70 & A I \\
        74 & A \\
        117 & A I \\
        121 & A II \\
        125 & A I \\
        129 & A II \\
        133 & A I \\
        137 & A II \\
        141 & A I \\
        145 & A II \\
        149 & A III* \\
        183 & A I \\
        
    \end{tabular}
    \caption{Tempo sequence in measures 1-183}
    \label{tab:first-tempo-sequence}
\end{table}

\subsection{Free and Strict Passages}

Aside from the main binary structure of fast versus slow tempo, the main logic whereby Barraqué creates formal distinction is by the juxtaposition of passages composed in ``strict''\footnote{\guillemotleft rigoureux\guillemotright} adherence to the serial system\footnote{\citet[39]{barraque-griffiths}} and passages composed more ``freely.''\footnote{\guillemotleft libre\guillemotright}

\begin{quote}
    \singlespacing
    ``It is appropriate to contrast a free style (for example, the beginning of the work) with a rigorous style. In rigorous passages (where subgroups of tempos must be established in the most precise way) a series of tempos and nuances are assigned to certain areas of the discourse; within these kinds of currents, indications of speed, dynamics and attacks come to contradict, cancel or amplify (see note on page 6) the general indication. We must therefore consider these notations as relative to the context.''\footnote{\citet{barraque-score}}
\end{quote}

Often, the strict passages can be visually discerned through the use of a boxed dynamic indication above the staff such as the \boxed{\lilyDynamics{p}} at measure 55. These strict passages develop Barraqué's serialized rhythm, octave, dynamic, and articulation processes, in contrast to the free passages which use only the pitch series and arbitrarily selected rhythm cells.\footnote{\citet[41]{barraque-griffiths}} Barraqué elaborates on this style dualism in a 1969 interview, saying:

\begin{quote}
    \singlespacing
    ``In the free style the greatest role is played effectively by the dynamics and by a sort of rhythmic élan that sets up very striking contrasts. On the other hand, the rigorous style is written in a very contrapuntal manner, where cells of the basic structure are developed according to a principle of variation I call `in closed-open circuit', that is, with all the variations on the rhythmic schemes, which are sometimes superimposed in two, three, four — up to four and even five parts — and which, above all, require the integration of silence, which progressively impregnates the work, so to say, and finally removes from it its contrapuntal and structural content to give way to silences — what I call `avoided' music, silences that have an importance in the work.''\footnote{\citet[40]{barraque-griffiths}}
\end{quote}

Henrich locates the sections of strict and free composition.\footnote{\citet[45]{barraque-werk}} Part one is divided as seen in table \ref{tab:part-1-styles}:

\begin{table}[H]
    \centering
    \begin{tabular}{c|l}
    style & location \\
    \hline
     & part 1  \\
    free & 1-54 \\
    strict & 55-73 \\
    free & 74-116 \\
    strict &  117-214 \\
    free & 215-247 \\
    strict & 248-417 \\
\end{tabular}
    \caption{Part 1 sections}
    \label{tab:part-1-styles}
\end{table}

Part two is divided as seen in table \ref{tab:part-2-styles}:

\begin{table}[H]
    \centering
    \begin{tabular}{c|l}
    style & location \\
    \hline
     & part 2  \\
    free & 418-562 \\
    strict & 563-669 \\
    free & 670-738 \\
\end{tabular}
    \caption{Part 2 sections}
    \label{tab:part-2-styles}
\end{table}


\subsection{Octavation}

By continuing to observe the strict passages at measure 55, it can be seen that two of the octave patterns are clearly deployed. The registration\footnote{That is, the octave at which each pitch class sounds.} changes at each tempo change alternating between pattern α' and β'. There is a slight inconsistency at the second appearance of α1 in the second system by the displacement of the B\mynatural. In a description of this process, Okuneva reveals that, in practice, C\mysharp \hspace{0.5mm} and F\mysharp \hspace{0.5mm} are voiced freely in any octave and that there is a loose coupling of tempo and octave arrangement.\footnote{\citet[2]{sonata-russia}} The verticalities found in Barraqué's sketches are as follows:

\subsubsection{Octavation α}

\begin{figure}[H]
    \includegraphics{lilypond/barraque/register_alpha.pdf}
    \caption{Octavation α}
    \label{fig:register-a}
\end{figure}

\begin{figure}[H]
    \includegraphics{lilypond/barraque/register_alpha_1.pdf}
    \caption{Octavation α'}
    \label{fig:register-a-1}
\end{figure}

\begin{figure}[H]
    \includegraphics{lilypond/barraque/register_alpha_2.pdf}
    \caption{Octavation α''}
    \label{fig:register-a-2}
\end{figure}

\subsubsection{Octavation β}

\begin{figure}[H]
    \includegraphics{lilypond/barraque/register_beta.pdf}
    \caption{Octavation β}
    \label{fig:register-b}
\end{figure}

\begin{figure}[H]
    \includegraphics{lilypond/barraque/register_beta_1.pdf}
    \caption{Octavation β'}
    \label{fig:register-b-1}
\end{figure}

\begin{figure}[H]
    \includegraphics{lilypond/barraque/register_beta_2.pdf}
    \caption{Octavation β''}
    \label{fig:register-b-2}
\end{figure}

\subsubsection{Octavation γ}

\begin{figure}[H]
    \includegraphics{lilypond/barraque/register_gamma.pdf}
    \caption{Octavation γ}
    \label{fig:register-g}
\end{figure}

\begin{figure}[H]
    \includegraphics{lilypond/barraque/register_gamma_2.pdf}
    \caption{Octavation γ''}
    \label{fig:register-g-2}
\end{figure}

These octavations were revised during the preparation of Yvonne Loriod's recording of the work.\footnote{\citet[75]{barraque-commentary}} The revised octavations are used in both the original publication and the urtext edition and were designed to support a greater ease of performance.

\section{Pitch}

Barraqué's harmonic language is consistent with the serial methodology. The composer defines a series of pitches which then undergoes various processes of manipulation including but not limited to the ``classical'' serial procedures of \emph{transposition}, \emph{inversion}, and \emph{retrograde}. The combination of one or more of these procedures produces a variety of row forms the composer may deploy within a work. The uniformity of the melodic/harmonic material produced by such a system, while often desirable,\footnote{Especially in brief works such as those produced by Anton Webern.} potentially creates a sense of monotony which many composers have attempted to avoid through the use of more than one basic tone-row within a work.

In compositions produced \emph{after} the \textit{Sonate pour piano},\footnote{For instance \textit{...au delà du hasard}} Barraqué develops a technique called ``proliferation'' whereby the ordering of elements within a row is permuted by the position of elements within another row.\footnote{\citet[20]{barraque-konzepte}} In figure \ref{fig:proliferation-1} each tone within a row is annotated with its position in the companion row. The pitches of each row are then reordered ascending by the index value of the companion row. In the case of this example, the first row is reordered where index 0 is labeled in position 1 in the second row. This position refers to A\myflat \hspace{0.5mm} in the first row, therefore the first tone in the newly constructed row is A\myflat. Index 1 in the companion row is found at position 5 which refers to D\mynatural \hspace{0.5mm} in the first row, therefore the second tone of the newly constructed row is D\mynatural. This process continues to produce a full tone row.\marginpar{Perhaps Barraqué developed this technique in response to some of Pierre Boulez's developments in row manipulation begun in \emph{Le Marteau sans maître} where the row no longer functions only as a motif but as a source from which other materials can be derived creatively.}

\begin{figure}[H]
    \includegraphics{lilypond/barraque/barraque_proliferation.pdf}
    \caption{Series proliferation}
    \label{fig:proliferation-1}
\end{figure}

This process of proliferation may be repeated to produce more row variations and Barraqué uses subtly different approaches in different works; however, in the \textit{Sonate pour piano}, Barraqué does not yet develop the proliferation technique,\footnote{Despite the absence of row proliferation in Barraqué's \textit{Sonate}, the technique is used in my composition \textit{Alu}.} instead allowing himself to occasionally permute the order of tones in the row.

\subsection{Pitch Series}

A matrix of all forms of the basic row of the work is illustrated in figure \ref{fig:matrix-a}. In the opening of the work, another binary opposition is at play. While most row statements are elided,\marginpar{Barraqué's sketches illustrate the various row forms in the fashion of René Leibowitz as two columns of prime and inverted rows with the retrogrades implied as reverse readings of the given rows. The rows are illustrated here as a matrix for its economical use of space.} Barraqué oscillates between prime and inverted forms of the row, beginning with a progression of \boxed{\text{T4}}, \boxed{\text{IT8}}, \boxed{\text{T5}}, \boxed{\text{IT11(r)}}, \boxed{\text{T9}}, \boxed{\text{IT3(r)}}, \boxed{\text{T0(r)}}, \boxed{\text{IT2}}, and \boxed{\text{T11}}.\footnote{The elision between \boxed{\text{IT11(r)}} and \boxed{\text{T9}} features some slight permutation of the ordering of their shared tetrachord.} Barraqué's choice of row forms in the opening of the piece is motivated primarily by his ability to link row statements by common tones. At this juncture there appears to be no pattern as to whether a given row form is read in retrograde or not. While the free passages do not make use of the strict passage octavation system, the rapidity of the wide registral changes of the opening of the work and the often loud dynamic levels contribute to the dramatic expressivity of the first half of the binary. Returning to the strict passage on page six, no such clear oscillation between prime and inverted rows appears, instead beginning with \boxed{\text{T0}}, \boxed{\text{T3}}, \boxed{\text{IT0(r)}}, and possibly \boxed{\text{IT3(r)}}.\footnote{The tones of \boxed{\text{IT3(r)}} appear to be permuted, however this progression suggest structural repetition through the duplication of the 0-3 transposition intervals.}

\begin{figure}[p]
\resizebox{\columnwidth}{!}{
    \includegraphics{lilypond/barraque/barraque_matrix.pdf}
    }
    \caption{Pitch series matrix}
    \label{fig:matrix-a}
\end{figure}

From the virtuosic analysis performed by Heribert Henrich, it can be seen that Barraqué eventually develops a more systematic approach to the choice of tone rows resulting in cyclic transposition sequences.\footnote{\citet[48]{barraque-werk}} A cycle begins at measure 42, sub-measure 3, starting at the quintuplet marked ``ritenuto'' through the end of measure 56 with repetitions of the row forms \boxed{\text{IT0}}, \boxed{\text{IT9}}, \boxed{\text{IT6}}, and \boxed{\text{IT3}}. These rows form a cycle because the final pitch of one row is the first pitch of the subsequent row, because of this feature \boxed{\text{IT3}} leads back to \boxed{\text{IT0}}. Henrich identifies further cyclic structures including row cycles formed from interleaved transposition cycles.\footnote{\citet[49]{barraque-werk}}. Measures 216-239 use a cycle of \boxed{\text{T11(r)}}, \boxed{\text{IT8(r)}}, \boxed{\text{T6(r)}}, \boxed{\text{IT3(r)}}, \boxed{\text{T1(r)}}, \boxed{\text{IT10(r)}}, \boxed{\text{T8(r)}}, \boxed{\text{IT5(r)}}, \boxed{\text{T3(r)}}, \boxed{\text{IT0(r)}}, \boxed{\text{T10(r)}}, \boxed{\text{IT7(r)}}, \boxed{\text{T5(r)}}, \boxed{\text{IT2(r)}}, \boxed{\text{T0(r)}}, \boxed{\text{IT9(r)}}, and \boxed{\text{T7(r)}}. In this cycle, inverted rows are related by transpositions of 7 semitones and prime rows also cycle through transpositions of 7 semitones.\footnote{\citet[49]{barraque-werk}}

Henrich notes that measure 397 begins the use of full meta-rows where subsequent row transpositions are related to the intervals of the row itself.\footnote{\citet[50]{barraque-werk}} This sequence is \boxed{\text{T0}}, \boxed{\text{T2}}, \boxed{\text{T7}}, \boxed{\text{T1}}, \boxed{\text{T9}}, \boxed{\text{T8}}, \boxed{\text{T10}}, \boxed{\text{T11}}, \boxed{\text{T5}}, \boxed{\text{T4}}, \boxed{\text{T6}}, and \boxed{\text{T3}} which corresponds to transpositions based on \boxed{\text{T0}}: \{0, 2, 7, 1, 9, 8, 10, 11, 5, 4, 6, 3\}.\footnote{Henrich suggests that this sequence is derived from \boxed{\text{T8}} as a result of his reference to Barraqué's rows by their Leibowitz numbers. The Leibowitz labels of this row cycle would be \{IX, XI, IV, X, VI, V, VII, VIII, II, I, III, XII\}. If C\mynatural \hspace{0.5mm} maps to the number 1, C\mysharp \hspace{0.5mm} maps to 2, etc. the macro-row corresponds to the pitch sequence \{A\myflat, B\myflat, E\myflat, A\mynatural, F\mynatural, E\mynatural, F\mysharp, G\mynatural, C\mysharp, C\mynatural, D\mynatural, B\mynatural\} which Henrich labels as \boxed{\text{V}}.} As seen from the above description of some of Barraqué's pitch processes in the \textit{Sonate}, structural contrast with pitch is available through the use of cyclic (or otherwise structured) sequences of row forms and freely chosen transpositions.

\section{Rhythm}

While I have had access to digitizations of some of Barraqué's sketches for the \textit{Sonate pour piano}, inconsistencies between the sketches and score led me to more heavily rely on Henrich's study of the rhythmic aspect of the work. While it is apparent that Barraqué derives rhythms in this work from a matrix of rhythm cells (as opposed to isolated, individual durations), exactly how they are deployed throughout the free passages is unclear. In their respective analyses, Okuneva and Hopkins illustrate different, though complementary, matrices which initially led to my deep misunderstanding of the rhythmic surface of the work.\footnote{\citet[20]{hopkins}} The technique bears similarity to techniques employed in Pierre Boulez's \textit{Deuxième sonate pour piano}\footnote{\citet[110]{boulez-piano}} with which Barraqué must have been familiar.\footnote{\citet[38]{barraque-griffiths}} As with other aspects of the \textit{Sonate}, contrast is produced throughout the piece by alternating between strict adherence to a rhythm scheme and free deployment of rhythmic creativity.

Barraqué first defines a series of eleven rhythm cells from which 156 more are derived, resulting in a total of 167 cells.

\begin{figure}[H]
    \includegraphics{lilypond/barraque/basic_motifs.pdf}
    \caption{Basic rhythm motifs}
    \label{fig:rhythm-modules}
\end{figure}

In the latter half of the piece, as the tempo slows down and the texture thins, rhythm cells are replaced by single note of the duration of an entire cell. For example, the cells in figure \ref{fig:rhythm-modules} sum to the values in figure \ref{fig:rhythm-sums}.

\begin{figure}[H]
    \includegraphics{lilypond/barraque/barraque_duration_sums.pdf}
    \caption{Sums of basic rhythm motifs}
    \label{fig:rhythm-sums}
\end{figure}

Barraqué manipulates the rhythm cells through a variety of techniques, both while generating the rhythm cells of the matrix for the strict sections as well as intuitively manipulating the rhythmic surface of the free sections, such as modifying the prolation of a cell, adding rests, reversing the cell, adding or removing grace notes, or tying notes together. This technique can be seen as early in the piece as the first two phrases. Phrase one, which occurs over the course of measures 1-3, comprises 12 modules as illustrated in figure \ref{fig:rhythm-mutation-1}. This example is derived from \citet[96]{barraque-konzepte} with various duration corrections coinciding with Henrich's urtext score.

In the subsequent phrase, from measures 4-7, Barraqué deploys the same sequence of cells with manipulations as seen in figure \ref{fig:rhythm-mutation-2} where most tuplets are exchanged for non-tuplet values and non-tuplet values are prolated. Some modules are more significantly modified such as module ten.

\begin{figure}[H]
    \includegraphics{lilypond/barraque/barraque_example_1.pdf}
    \caption{Basic rhythms of measures 1-3}
    \label{fig:rhythm-mutation-1}
\end{figure}

\begin{figure}[H]
    \includegraphics{lilypond/barraque/barraque_example_2.pdf}
    \caption{Basic rhythms of measures 4-7}
    \label{fig:rhythm-mutation-2}
\end{figure}

The rhythms are sequentially deployed within the strict passages. The first half of the piece uses only cells derived from motifs 1-6 while the second half of the piece uses cells based on all available motifs. In the first half of the work, sub-matrices are defined for each motif such that two series of variations on the same module run in parallel. The first strict section uses rhythms based on motif one and two, the second uses motifs three and four, and the third uses five and six.

Just as the octavations within this section are coupled with the tempo changes, so are the changes in source motif. Measures 55-56 show tempo AII associated with rhythms from motif one and tempo AI is coupled to motif 2. In the sketches, Barraqué defines two separate streams of cells for motif 1 and two streams for motif 2. These four series are labeled as \boxed{\text{1A}}, \boxed{\text{1a}}, \boxed{\text{2B}}, and \boxed{\text{2b}}. The A series feature twelve cells each and each B series comprises eleven cells. In this section, they are interleaved producing the sequence \{ IA1, Ia1, IA2, Ia2, IA3, Ia3, IA4\}, \{2B1, 2b1, 2B2, 2b2, 2B3, 2b3, 2b4, 2B4\}, \{1a4, 1A5, 1a51A6, 1a6, 1A7, 1a7, 1a8, 1A8\}, et cetera. 

\section{Dynamics}

\marginpar{In the strict passages, the dynamics feature nested layers of information just as with the tempo manipulation, however this results in contradictory information.}Henrich notes that the use of dynamics, while not entirely systematized shows some development across the strict passages.\footnote{\citet[53]{barraque-werk}} The first strict section (55-73) makes use of \{\lilyDynamics{pp}, \lilyDynamics{p}, \lilyDynamics{f}\}, the second (117-214) uses \{\lilyDynamics{p}, \lilyDynamics{mp}, \lilyDynamics{mf}, \lilyDynamics{f}\} the third (248-417) uses \{\lilyDynamics{pp}, \lilyDynamics{p}, \lilyDynamics{mf}, \lilyDynamics{f}, \lilyDynamics{ff}, \lilyDynamics{fff}\} and the final strict section (563-669) uses \{\lilyDynamics{pp}, \lilyDynamics{p}, \lilyDynamics{mp}, \lilyDynamics{mf}, \lilyDynamics{f}, \lilyDynamics{ff}, \lilyDynamics{fff}\}. This produces a gradual increase in the available dynamics for each section. As has been seen in other parametric features of the strict passages, a global dynamic is typically coupled with a particular tempo, octavation, and rhythm series.

\section{Articulations}

In his sketches for the \textit{Sonate}, Barraqué considered the use of two series of articulations. Despite the elaborate use of a variety of articulations throughout the piece, the systematic articulations were not fully realized in the final work. Henrich does note a correspondence in the second strict section (measures 117-214) where the articulations, freely applied to the next note of the composer's choosing, seem to almost follow the sequence of articulations found in the sketches.\footnote{\citet[54]{barraque-werk}} Each group of articulations would apply to the notes of a single measure.

\subsection{Series a}

Accent series A comprises four groups of four patterns, each of which concatenates three to six elements. 

\begin{table}[H]
    \centering
    \begin{tabular}{c | c c c c}
        Series A & & & & \\
        \hline
         1 & \boxed{\text{\tenuto \hspace{2mm} \lilyStaccato \hspace{2mm} \lilyStaccato \hspace{2mm} \tenuto}} & \boxed{\text{\lilyAccent \hspace{2mm} \tenuto \hspace{2mm} \lilyStaccato \hspace{2mm} \lilyAccent}}  & \boxed{\text{\portato \hspace{2mm} \tenuto \hspace{2mm} \lilyGlyph{ties.lyric.default}\hspace{0.5mm}\lilyStaccato}} & \boxed{\text{\lilyStaccato \hspace{2mm} \lilyStaccato \hspace{2mm} \lilyStaccato \hspace{2mm} \portatoDown}} \\
         2 & \boxed{\text{\hspace{2mm} \lilyGlyph{ties.lyric.default}\hspace{0.5mm} \lilyStaccato \hspace{2mm} \lilyStaccato}} & \boxed{\text{\tenuto \hspace{2mm} \lilyStaccato\lilyGlyph{scripts.sforzato} \hspace{2mm} \tenuto \hspace{2mm} \tenuto}} & \boxed{\text{\lilyStaccato \hspace{2mm} \lilyAccent \hspace{2mm} \lilyGlyph{ties.lyric.default}\hspace{0.5mm}\lilyStaccato \hspace{2mm} \lilyGlyph{ties.lyric.default}\hspace{2mm} \tenuto}} & \boxed{\text{\tenuto \hspace{2mm} \lilyAccent \hspace{2mm} \lilyStaccato\lilyGlyph{scripts.sforzato} \hspace{2mm} \lilyStaccato\lilyGlyph{scripts.sforzato} \hspace{2mm} \lilyStaccato \hspace{2mm} \lilyAccent}} \\
         3 & \boxed{\text{\lilyGlyph{ties.lyric.default}\hspace{0.5mm}\lilyStaccato \hspace{2mm} \lilyStaccato \hspace{2mm} \lilyGlyph{ties.lyric.default}\hspace{2mm} \lilyStaccato}} & \boxed{\text{\lilyAccent \hspace{2mm} \lilyStaccato \hspace{2mm} \lilyAccent}} & \boxed{\text{\lilyStaccato \hspace{2mm} \lilyStaccato\lilyGlyph{scripts.sforzato} \hspace{2mm} \tenuto}} & \boxed{\text{\lilyStaccato \hspace{0.5mm} \lilyGlyph{ties.lyric.default} \hspace{0.25mm} \tenuto \hspace{2mm} \tenuto}} \\
         4 & \boxed{\text{\lilyStaccato \hspace{2mm} \tenuto \hspace{2mm} \lilyStaccato \hspace{2mm} \lilyStaccato\lilyGlyph{scripts.sforzato} \hspace{2mm} \lilyStaccato}} & \boxed{\text{\lilyStaccato \hspace{2mm} \tenuto \hspace{2mm} \portatoDown \hspace{2mm} \portatoDown \hspace{2mm} \tenuto}} & \boxed{\text{\tenuto \hspace{2mm} \marcatoDown \hspace{2mm} \lilyStaccato\lilyGlyph{scripts.sforzato} \hspace{2mm} (sp)}} & \boxed{\text{\lilyGlyph{ties.lyric.default}\hspace{0.5mm}\lilyStaccato \hspace{2mm} \lilyGlyph{ties.lyric.default}\hspace{0.5mm}\lilyStaccato \hspace{2mm} \marcatoDown}}
    \end{tabular}
    \caption{Diplomatic transcription of accent series A}
    \label{tab:barraque-accent-a}
\end{table}

\subsection{Series b}

Likewise, accent series B also comprises four groups of four patterns, each of which concatenates three to six elements. A relationship is established between each series such that series B is almost, but not precisely, a retrograde of series A.
\begin{table}[H]
    \centering
    \begin{tabular}{c | c c c c}
        Series B & & & & \\
        \hline
         1 & \boxed{\text{\portatoDown \hspace{2mm} \lilyStaccato\lilyGlyph{ties.lyric.default} \hspace{2mm} \lilyStaccato\lilyGlyph{ties.lyric.default}}} & \boxed{\text{\lilyStaccato\lilyGlyph{scripts.sforzato} \hspace{2mm} sp\hspace{0.25mm}\tenuto \hspace{2mm} \marcatoDown \hspace{2mm} \tenuto}}  & \boxed{\text{\tenuto \hspace{2mm} \portatoDown \hspace{2mm} \portatoDown \hspace{2mm} \tenuto \hspace{2mm} \lilyStaccato}} & \boxed{\text{\lilyStaccato \hspace{2mm} \lilyStaccato\lilyGlyph{scripts.sforzato} \hspace{2mm} \lilyStaccato \hspace{2mm} \tenuto \hspace{2mm} \lilyStaccato}} \\
         2 & \boxed{\text{\portato \hspace{2mm} \tenuto \hspace{0.5mm} \lilyGlyph{ties.lyric.default} \hspace{0.25mm} \lilyStaccato}} & \boxed{\text{\tenuto \hspace{2mm} \lilyStaccato\lilyGlyph{scripts.sforzato} \hspace{2mm} \lilyStaccato}} & \boxed{\text{\lilyAccent \hspace{2mm} \lilyStaccato \hspace{2mm} \lilyAccent}} & \boxed{\text{\lilyStaccato \hspace{2mm} \lilyStaccato\hspace{0.5mm}\lilyGlyph{ties.lyric.default} \hspace{2mm} \lilyStaccato \hspace{2mm} \lilyGlyph{ties.lyric.default}}} \\
         3 & \boxed{\text{\lilyAccent \hspace{2mm} \lilyStaccato \hspace{2mm} \lilyStaccato\lilyGlyph{scripts.sforzato} \hspace{2mm} \lilyStaccato\lilyGlyph{scripts.sforzato}\hspace{2mm} \lilyAccent \hspace{2mm} \tenuto}} & \boxed{\text{\tenuto \hspace{2mm} \lilyGlyph{ties.lyric.default} \hspace{2mm} \lilyStaccato\hspace{0.5mm}\lilyGlyph{ties.lyric.default} \hspace{2mm} \lilyAccent \hspace{2mm} \lilyStaccato}} & \boxed{\text{\tenuto \hspace{2mm} \tenuto \hspace{2mm} \lilyStaccato\lilyGlyph{scripts.sforzato} \hspace{2mm} \tenuto}} & \boxed{\text{\lilyStaccato \hspace{2mm} \lilyStaccato \hspace{2mm} \lilyGlyph{scripts.sforzato}}} \\
         4 & \boxed{\text{\portatoDown \hspace{2mm} \lilyStaccato \hspace{2mm} \lilyStaccato \hspace{2mm} \lilyStaccato}} & \boxed{\text{\lilyStaccato\hspace{0.5mm}\lilyGlyph{ties.lyric.default} \hspace{2mm} \tenuto \hspace{2mm} \portatoDown}} & \boxed{\text{\lilyAccent \hspace{2mm} \lilyStaccato \hspace{2mm} \tenuto \hspace{2mm} \lilyAccent}} & \boxed{\text{\tenuto \hspace{2mm} \lilyStaccato \hspace{2mm} \lilyStaccato \hspace{2mm} \tenuto}}
    \end{tabular}
    \caption{Diplomatic transcription of accent series B}
    \label{tab:barraque-accent-b}
\end{table}

\section{Conclusion and Personal Application}

Here we leave the work for a future, more detailed analysis, not because I do not find Barraqué's sonata convincing,\marginpar{In fact, I hold this piece in incredibly high regard.} rather because it stands as an excellent example of the kind of large-scale formal structures which I find unfulfilling. While Barraqué deploys unique orderings of various parametric series within this work, the lack of clear material partitioning or parametric coupling nevertheless produces, in my listening, a sense of uniformity. Coupling does occur during the strict passages; however, in my listening it is the octavation and tempo which provide the clearest sectionality as opposed to the rhythm, pitch or dynamic aspects of these sections. It is probably the decoupling of pitch from rhythm which renders the visual clarity of the rhythm modules relatively occluded in sound. This occlusion is amplified in the passages with fixed registrations where the clearly, visually identifiable rhythmic motifs are practically indistinguishable since they are not contrasted in dynamic, register, harmony, or articulation.

While the work is gravid with contrastive potential (e.g. Barraqué's deep investment in dualisms), the changes between free and strict passages function as no more than slight changes of color of the same basic idea. Even Barraqué's deployment of tempo change is tepid, lacking the dramatic confidence of his use of wide registral sweeps within the opening gestures.%\footnote{Despite Klaus Linder's analysis to the contrary.} 

The shape of the overall form is also incredibly Romantic or even Galant in its scope. The passages of free composition versus strict composition, fast tempo versus slow tempo are of such a large scale that one does not feel them rubbing against one another, fighting for dominance or even providing a setting for discourse. Instead, the effect is one long rallantando of undifferentiated mono-material. Structural comparisons can be drawn between this work and more traditional sonata structures. Another fruitful comparison could be to the generic A-B-A' struture of Brian Ferneyhough's \textit{Unity Capsule}.

The most exciting aspect of the work for me is rhythmic dynamism produced by the figural quality of the rhythmic cells and the collaboration of the pitch material with the rhythms to produce recognizable gestures in the free passages. Since much of the rhythmic components of my own works are derived from textural thinking, I am encouraged to pursue a more figural language in future works.\marginpar{In fact, this journey has already begun in compositions such as my \emph{Torlannol} or \emph{Infiorescenze}.} While analyzing this piece, I find a work which is both beautiful and emotionally dynamic, however my formal experience is not ideal. It is very difficult for me to follow the changes throughout the work, therefore I think an approach with greater material distinction will provide the solution I need to articulate form.

In the above analysis, the concept of material is primarily parametric. The twelve-tone language, the rhythm cells, the octavation, articulations, and dynamics are all materials. These materials are grouped into events not so much characterized by the presence or absence of certain materials, rather the events are characterized by the manner through which the composer conceives of the music. This work, then, is not a strong example of formalized event groups or formal variegation; however, what can be seen is the ethos of Rational Thaumaturgy. While a kind of unity is provided by the consistent use of a basic twelve-tone motif, Barraqué freely deploys many approaches to organization throughout the work, suggesting the purpose of formalization is not a pseudo-scientific desire for perfection. The organization provides a cipher discovered under analysis which might not otherwise be perceived. The arc of the sonata is one of increasing rigidity and starkness where the composition finally disintegrates under the pressure of its own limitations. Barraqué's use of various compositional methodologies potentially provides a meta-commentary on his feelings of composing in modernity.