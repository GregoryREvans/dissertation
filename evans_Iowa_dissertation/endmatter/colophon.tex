\newpage

% If you do want an image in the colophon:
\begin{figure}
  \vspace{50pt}
  \centering
    \includegraphics[width=0.51\textwidth]{figures/white_skull.png}
\end{figure}

% If you don't want an image in the colophon:
% \vspace*{200pt}

\begin{center}
\parbox{0.51\textwidth}{\lettrine[lines=3,slope=-2pt,nindent=-4pt]{\textcolor{SchoolColor}{T}}{his dissertation} was typeset using \hologo{XeLaTeX}, which includes extensions to Jonathan Kew's \hologo{XeTeX} \ executable to allow Unicode input and loads the \LaTeX \ format. \LaTeX \ was developed by Leslie Lamport and based on Donald Knuth’s \TeX. The body text is set in 12 point Egenolff-Berner Garamond, a revival of Claude Garamont's humanist typeface. The initial text at the beginning of each chapter is set in Goudy Initialen. The above illustration, ``A Skull'', was created by Andrea Andreani in 1588 and is in the public domain. The frontispiece illustration is ``Der Zauberlehrling'' from \textit{Göthe's Werke} (1882), drawn by Ferdinand Barth (1842–1892) and is in the public domain. All other plates are also in the public domain. All music notation graphics were created by the author with Lilypond. Reproductions of Trevor Ba\v{c}a's scores and sketches was done so with permission from the composer. All source code listings are from open-source software. A template that can be used to format a PhD thesis with this look and feel has been released under the permissive \textsc{mit} (\textsc{x}11) license, and can be found online at \href{https://github.com/suchow/Dissertate}{github.com/suchow/Dissertate} or from its author, Jordan Suchow, at \href{mailto:suchow@post.harvard.edu}{suchow@post.harvard.edu}.}
\end{center}