% the scholarly abstract

While developing my compositional practice over the last five years, I was influenced to innovate on my techniques because of my contact with the music of several key composers and through the reevaluation of musical modernism and large-scale musical form. Through current scholarship, I understand musical modernism as a broader artistic movement than is commonly misunderstood. Musical modernism is often taught as an un-intuitive practice where the composer designs an automaton which generates the entirety of a composition or that a single musical object, such as a twelve-tone row, is the sole source of the derivation of all musical elements within a work.  Notions of a post-modern or post-serial music are often drawn from lack of understanding of the actual openness within the categories which they intend to negate. While some modernists present a rigorous and dogmatic philosophy, it is clear to me that it is necessary to categorize artists who use structure outside of a hegemonic framework. To describe this orientation, I take on a metaphor of magic or enciphering with the term ``Rational Thaumaturgy.'' In preparing the analytic chapters of this dissertation, I undertook a serious intervention on my thinking about musical form. By broadening the content of analysis beyond traditional concepts of themes, harmony, or motivic development to the more abstract concept of ``material,'' certain subtle relationships between musical moments across the timeline of a composition can be understood. Importantly, the concept of musical ``materials'' are often intentionally deployed by composers. To present my argument for both Rational Thaumaturgy and formal polyphony, or ``variegated form,'' I performed analysis on works by several composers whose work has been highly influential on my own yet have little or no published scholarship in the English language or no published scholarship of any kind: Jean Barraqué's \textit{Sonate pour piano}, Francsico Guerrero Marín's \textit{Rhea} for 12 saxophones, and Trevor Bača's \textit{Akasha} for string quartet. To analyze these works, I familiarized myself with non-English scholarship when it was available and I consulted primary-source sketch material. After identifying important features of these works, such as aspects of their large-scale form or their relationship to ideas of formalization, I describe two of my own works, \textit{Polillas} for string quartet and \textit{Alu} for sinfonietta, which share intentional similarities with the analyzed works.