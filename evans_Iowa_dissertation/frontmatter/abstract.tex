% the public abstract

In this dissertation, I describe certain developments within my compositional practice and the origins of these techniques in the works \textit{Sonate pour piano} by Jean Barraqué, \textit{Rhea} for 12 saxophones by Francisco Guerrero Marín, and \textit{Akasha} for string quartet by Trevor Bača. By reassessing common assumptions about the æsthetics of musical modernism and developing new techniques for analysing pieces which do not strictly adhere to a single compositional technique, I develop a personal musical language which embraces the admiration for structural coherence found in many modernist compositions while also allowing for freedom in compositional decision-making and allusions to traditional notions of beauty. Barraqué's \textit{Sonate} is designed as a structural opposition between freedom and rigor. In this work contrast is provided not by the basic materials employed in each section, rather it is the compositional approach used by the composer which is the source of juxtaposition. Guerrero's \textit{Rhea} produces contrast through the juxtaposition of seven highly unique, precisely defined musical materials. Guerrero takes a quasi-mathematical approach to the composition of most of his works, including \textit{Rhea}, however this work also reveals an underlying binary structure where the contrapuntal distribution of materials is repeated with new materials replacing the old ones. Guerrero also controls dramatic intensity through the density of the material counterpoint. Bača's \textit{Akasha} deploys a type of formal structure I define as ``variegation'' where simultaneous statements of musical ideas are heard as independent streams of information, not merely contributions to one mass texture. Bača's approach to rhythm is highly systematized and is often able to be used as a distinguishing feature for each musical material. My own works \textit{Polillas} for string quartet and \textit{Alu} for sinfonietta use many of the same conceptual frameworks as the previously mentioned compositions. I aim for both systematized stricture and free decision-making as I compose; I produce contrast through the definition of unique materials; and I systematize aspects of rhythm and harmony in my work. I see my compositional practice as an extension of musical modernism, not as post-modernism, due to the realization that modernism is not a monolithic ideology; rather, it is a constellation of many, sometimes contradictory, ideas. Within the broad heading of modernism, I define the term ``Rational Thaumaturgy'' to refer to works which deploy aspects of formalization without the dogmatic aim of a completely rigorous or unified whole. The term is applicable not only to my own music, but also to the previously mentioned works by Barraqué and Bača.